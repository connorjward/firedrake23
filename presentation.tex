\documentclass[aspectratio=169]{beamer}

% packages
\usepackage{graphicx}
  \graphicspath{{figures}}
\usepackage{minted}
\usepackage{amssymb}
\usepackage{tcolorbox}
\usepackage{xcolor}

% subfig support
\usepackage{caption}
\usepackage{subcaption}

% packages
\usepackage{biblatex}
  \addbibresource{bibliography.bib}
\usepackage[acronym,nomain]{glossaries}
  \setacronymstyle{long-short}
  \newacronym[shortplural={DoFs},longplural={degrees-of-freedom}]
    {dof}{DoF}{degree-of-freedom}
  \newacronym{fem}{FEM}{the finite element method}
  \newacronym{pde}{PDE}{partial differential equation}
  \newacronym[shortplural={FLOPs},longplural={floating-point operations}]
    {flop}{FLOP}{floating-point operation}
  \newacronym{dag}{DAG}{directed acyclic graph}
  \newacronym{dg}{DG}{discontinous Galerkin}
  \newacronym{poset}{poset}{partially-ordered set}
  \newacronym{rcm}{RCM}{reverse Cuthill-McKee}
  \newacronym{dsl}{DSL}{domain-specific language}
  \newacronym{jit}{JIT}{just-in-time}
  \newacronym{ufl}{UFL}{the Unified Form Language}
  \newacronym{tsfc}{TSFC}{the Two-Stage Form Compiler}
\usepackage{graphicx}
  \graphicspath{{figures}}
\usepackage{todonotes}
\usepackage{hyperref}
\usepackage{subcaption}
\usepackage{amsmath}
% source: https://tex.stackexchange.com/questions/650034/mathbb-font-for-lowercase-letters
\usepackage[bb=libus]{mathalpha}
\usepackage{pgf}
\usepackage{pgfplots}
\usepackage{tikz}
\usepackage{tkz-euclide}
  % \usetikzlibrary{arrows,calc,graphs,graphdrawing,positioning,tikzmark,shapes.geometric,patterns.meta,decorations.pathreplacing,decorations.marking}
  \usetikzlibrary{arrows,calc,graphs,graphdrawing,positioning,tikzmark,shapes.geometric,patterns.meta,decorations.pathreplacing}
  \usegdlibrary{trees}
  \pgfdeclarelayer{background}
  \pgfsetlayers{background,main}
  \tikzstyle{ptlabel} = [anchor=center, color=black, opacity=1]
  \tikzset{font={\small}}
  \tikzset{label style/.append style={font=\small}}
  % source: https://tex.stackexchange.com/questions/356564/macro-for-rounded-polygon-around-some-nodes
  \def\drawpolygon#1,#2;{
    \begin{pgfonlayer}{background}
        \filldraw[line width=28,join=round](#1.center)foreach\A in{#2}{--(\A.center)}--cycle;
        \filldraw[line width=27,join=round,blue!10](#1.center)foreach\A in{#2}{--(\A.center)}--cycle;
    \end{pgfonlayer}
  }

% theme info
\usetheme{firedrake}

% title info
\title{Latest developments in \texttt{pyop3}}
\author{Connor Ward}
\date{13 September 2023}

% macros

% checkbox
% source https://tex.stackexchange.com/questions/16000/creating-boxed-check-mark
\newcommand{\unchecked}{\makebox[0pt][l]{$\square$}\raisebox{.15ex}{\hspace{0.1em}$\quad$}}
\newcommand{\maybe}{\makebox[0pt][l]{$\square$}\raisebox{.15ex}{\hspace{0.1em}$\lozenge$}}
\newcommand{\checked}{\makebox[0pt][l]{$\square$}\raisebox{.15ex}{\hspace{0.1em}$\checkmark$}}

% hacky way to get \pyop2 and \pyop3 as valid macros
% source: https://tex.stackexchange.com/questions/13290/how-to-define-macros-with-numbers-in-them
\def\pyop#1{\ifnum#1=2 {PyOP2}\else \ifnum#1=3 {\texttt{pyop3}}\fi \fi}

\newcommand{\py}{\mintinline{python}}
\newcommand{\clang}{\mintinline{c}}
\newcommand{\closure}{\mathbb{cl}}
\newcommand{\support}{\mathbb{supp}}
\newcommand{\plexstar}{\mathbb{st}}
\newcommand{\cone}{\mathbb{cone}}

\newcommand{\hdiv}{$H(\mathrm{div})$ }
\newcommand{\hcurl}{$H(\mathrm{curl})$ }

\tikzstyle{plexstencil} = [draw=none,fill=blue!30,fill opacity=.4]

\newcommand{\codebgcolor}{black!10}

\newcommand{\myred}[1]{\textcolor{red}{\textbf{#1}}}

% set default parameters for tcolorbox
\tcbset{
  left=2mm
}

% set default minted parameters
\setminted{fontsize=\footnotesize}

% =============================================================================

\begin{document}

\frame{\titlepage}

\begin{frame}{Overview}
  \tableofcontents
\end{frame}

\section{What is \pyop3?}

\begin{frame}{TL;DR}
  \begin{itemize}
    \item A programming language for mathematicians
    \item Comes with a compiler
    \item The language lets you express how to read and write from complicated data structures
  \end{itemize}

  \vfill
\end{frame}

\begin{frame}{In more detail}
  \begin{itemize}
    \item A \myred{domain-specific} programming language for mathematicians \myred{embedded in Python}
    \item Comes with a \myred{just-in-time} compiler \myred{that targets loopy and then C/CUDA/OpenCL}
    \item The language lets you express how to read and write from complicated data structures
    \item \myred{Never need to create a \texttt{PetscSection} ever again!}
  \end{itemize}

  \vfill
\end{frame}

\begin{frame}{Why is this hard?}
  \begin{itemize}
    \item FEM codes have diverse and complicated data structures
    \item These data structures also need to be accessed in non-trivial ways
  \end{itemize}
\end{frame}

\section{A simple-ish example}

\begin{frame}[fragile]{Creating a data layout for a mesh}
  \noindent
  \begin{minipage}{.4\textwidth}
    \begin{tikzpicture}[scale=.9]
      \tkzDefPoint(0,2){v0}
\tkzDefPoint(2,4){v1}
\tkzDefPoint(2,0){v2}
\tkzDefPoint(4,2){v3}

\tkzDrawSegments(v0,v1 v0,v2 v1,v2 v1,v3 v2,v3)
\tkzDrawPoints(v0,v1,v2,v3)

\tkzDefBarycentricPoint(v0=1.2,v1=1,v2=1) \tkzGetPoint{c0}
\tkzLabelPoint[centered,xshift=-.1cm](c0){3}
\tkzDefBarycentricPoint(v1=1,v2=1,v3=1.2) \tkzGetPoint{c1}
\tkzLabelPoint[centered,xshift=.1cm](c1){7}

\tkzLabelPoint[left](v0){1}
\tkzLabelPoint[above](v1){6}
\tkzLabelPoint[below](v2){4}
\tkzLabelPoint[right](v3){8}

\tkzLabelSegment[below left](v0,v2){0}
\tkzLabelSegment[above left](v0,v1){10}
\tkzLabelSegment[left,xshift=.1cm](v1,v2){5}
\tkzLabelSegment[above right](v1,v3){9}
\tkzLabelSegment[below right](v2,v3){2}

    \end{tikzpicture}
  \end{minipage}%
  \begin{minipage}{.55\textwidth}
    \begin{tcolorbox}
      \begin{minted}[fontsize=\tiny]{python}
mesh_axis = Axis(
  [
    AxisComponent(2, "cells"),
    AxisComponent(5, "edges"),
    AxisComponent(4, "verts"),
  ],
  "mesh"
  permutation=[3, 7, 0, 10, 5, 9, 2, 1, 6, 4, 8],
)
      \end{minted}
    \end{tcolorbox}
  \end{minipage}

  \pause
  \vspace{-2em}

  \begin{center}
    \includegraphics{scripts/two_cell_mesh.gv.pdf}
  \end{center}
\end{frame}

\begin{frame}[fragile]{Now make it P3}
  \noindent
  \begin{minipage}{.4\textwidth}
    \begin{tikzpicture}[scale=1.5]
      \input{lagrange3.tikz}
    \end{tikzpicture}
  \end{minipage}%
  \begin{minipage}{.55\textwidth}
    \begin{tcolorbox}
      \begin{minted}[fontsize=\tiny]{python}
axes = (
  AxisTree(mesh_axis)
    .add_subaxis(Axis(1), mesh_axis.id, "cells")
    .add_subaxis(Axis(2), mesh_axis.id, "edges")
    .add_subaxis(Axis(1), mesh_axis.id, "verts")
)
      \end{minted}
    \end{tcolorbox}
  \end{minipage}

  \pause
  \vspace{-2em}

  \begin{center}
    \includegraphics[width=\textwidth]{scripts/two_cell_P3_layout.gv.pdf}
  \end{center}
\end{frame}

\begin{frame}{A typical non-trivial data access pattern: residual assembly}
  \begin{tikzpicture}
    \input{stencil_computation.tikz}
  \end{tikzpicture}
\end{frame}

\begin{frame}[fragile]{Residual assembly in \pyop3}
  \begin{tcolorbox}
    \begin{minted}{text}
for every cell in the mesh:
  collect DoFs found in the cell's closure
  call a local kernel with these DoFs
  scatter the result to a global vector
    \end{minted}
  \end{tcolorbox}

  \pause

  \begin{tcolorbox}
    \begin{minted}{python}
loop(
  c := mesh.cells.index(),
  kernel(func0[closure(c)], ...)
)
    \end{minted}
  \end{tcolorbox}
\end{frame}

\begin{frame}[fragile]{Code generation!}
  \begin{tcolorbox}
    \begin{minted}[fontsize=\tiny]{c}
void my_loop(double *func0, int *map0, int *map1, int *layout0, int *layout1, int *layout2, ...) {
  // to store the "packed" data
  double t_0[10];
  // loop over cells
  for (int32_t i_0 = 0; i_0 < 2; ++i_0) {
    // pack cell DoFs
    t_0[0] = func0[layout0[i_0]];
    // pack edge DoFs
    for (int32_t i_5 = 0; i_5 < 3; ++i_5) {  // loop over edges
      for (int32_t i_6 = 0; i_6 < 2; ++i_6) {  // loop over edge DoFs
        j_3 = map0[i_0 * 3 + i_5];  // select the right edge
        t_0[i_5*2 + i_6 + 1] = func0[layout1[j_3] + i_6];  // pack DoF
      }
    }
    // pack vertex DoFs
    for (int32_t i_7 = 0; i_7 < 3; ++i_7) {  // loop over vertices
      j_5 = map1[i_0 * 3 + i_7];  // select the right vertex
      t_0[i_7 + 7] = func0[layout2[j_5]];  // pack DoF
    }
    // execute the local kernel
    kernel(t_0, ...);
    // now unpack the result in the same way
  }
}
    \end{minted}
  \end{tcolorbox}
\end{frame}

% TODO:
% * Now I've explained how it works I should demonstrate cooler bits
% * Ragged/particle data layouts, p-adaptivity?
% * Loop/map composition
% * Should emphasise composability of data layouts and loop expressions!!!

\end{document}
